\documentclass{article}

\usepackage{fontspec}

% APL unicode font
\setmonofont{APL385}

% smaller margins in document 
\usepackage[headings]{fullpage}

% extended coloring
\usepackage[dvipsnames]{xcolor}

\definecolor{CodeBackGround}{cmyk}{0.0,0.0,0,0.05}  % light gray
\definecolor{CodeComment}{rgb}{0,0.50,0.00}         % dark green {0,0.45,0.08}

\usepackage{listings}

% use long form control keyword terminators, i.e. endif 
% instead of end.  It's a language design flaw to
% support both forms. APL+WIN makes this mistake.
\lstdefinelanguage{apl}
{
morekeywords={},
otherkeywords={:if,:else,:for,:in,:while,:until
,:andif,:orif,:repeat,:select,:case
,:go,:return,:try,:catch,:catchif,:catchall
,:endfor,:endwhile,:endselect,:endif,:endrepeat,:endselect},
keywordstyle=\bfseries\color{red},
sensitive=True,
morecomment=[l]{⍝}
}

\lstset{%
  basicstyle=\ttfamily\small,  
  keywordstyle=\bfseries\normalsize,  % can color key words        
  identifierstyle=,                             
  commentstyle=\slshape\color{CodeComment}, % no slanted shape in APL385    
  stringstyle=\ttfamily,                        
  showstringspaces=false,                      
  backgroundcolor=\color{CodeBackGround},      
  frame=single,                               
  framesep=1pt,                                
  framerule=0.8pt,                              
  rulecolor=\color{CodeBackGround},             
  breaklines=true,      % break long code lines                        
  breakindent=0pt                               
}

\makeatletter
\lst@InputCatcodes
\def\lst@DefEC{%
 \lst@CCECUse \lst@ProcessLetter
  ^^80^^81^^82^^83^^84^^85^^86^^87^^88^^89^^8a^^8b^^8c^^8d^^8e^^8f%
  ^^90^^91^^92^^93^^94^^95^^96^^97^^98^^99^^9a^^9b^^9c^^9d^^9e^^9f%
  ^^a0^^a1^^a2^^a3^^a4^^a5^^a6^^a7^^a8^^a9^^aa^^ab^^ac^^ad^^ae^^af%
  ^^b0^^b1^^b2^^b3^^b4^^b5^^b6^^b7^^b8^^b9^^ba^^bb^^bc^^bd^^be^^bf%
  ^^c0^^c1^^c2^^c3^^c4^^c5^^c6^^c7^^c8^^c9^^ca^^cb^^cc^^cd^^ce^^cf%
  ^^d0^^d1^^d2^^d3^^d4^^d5^^d6^^d7^^d8^^d9^^da^^db^^dc^^dd^^de^^df%
  ^^e0^^e1^^e2^^e3^^e4^^e5^^e6^^e7^^e8^^e9^^ea^^eb^^ec^^ed^^ee^^ef%
  ^^f0^^f1^^f2^^f3^^f4^^f5^^f6^^f7^^f8^^f9^^fa^^fb^^fc^^fd^^fe^^ff%
  ^^^^20ac^^^^0153^^^^0152%
  ^^^^20a7^^^^2190^^^^2191^^^^2192^^^^2193^^^^2206^^^^2207^^^^220a%
  ^^^^2218^^^^2228^^^^2229^^^^222a^^^^2235^^^^223c^^^^2260^^^^2261%
  ^^^^2262^^^^2264^^^^2265^^^^2282^^^^2283^^^^2296^^^^22a2^^^^22a3%
  ^^^^22a4^^^^22a5^^^^22c4^^^^2308^^^^230a^^^^2336^^^^2337^^^^2339%
  ^^^^233b^^^^233d^^^^233f^^^^2340^^^^2342^^^^2347^^^^2348^^^^2349%
  ^^^^234b^^^^234e^^^^2350^^^^2352^^^^2355^^^^2357^^^^2359^^^^235d%
  ^^^^235e^^^^235f^^^^2361^^^^2362^^^^2363^^^^2364^^^^2365^^^^2368%
  ^^^^236a^^^^236b^^^^236c^^^^2371^^^^2372^^^^2373^^^^2374^^^^2375%
  ^^^^2377^^^^2378^^^^237a^^^^2395^^^^25af^^^^25ca^^^^25cb%  
  ^^00}
\lst@RestoreCatcodes
\makeatother

\begin{document}

Plain verbatim listing. It works for UTF8 input but does not break
lines, format syntax or update listings list.

\begin{verbatim}
z ← keyn SQLiteFrDBI dbi;t;n;st;sn;sc;sql;rf;rm;nrc;⎕io

⍝ Returns a SQLite table creation statement that maps the
⍝ structure of APL+WIN inverted DBI files to SQLite tables.
⍝
⍝ monad: ev ← SQLiteFrDBI cvDBIFileName
⍝
⍝   sqlv ← SQLiteFrDBI 'C:\BCA\bcadev\CA\ULTCL.DBI'

 ⎕io ← 1

⍝ default SQLite primary key name
 :if 0=⎕nc 'keyn' ⋄ keyn ← 'rwkey' ⋄ :endif

⍝ open DBI file - the DBI open function creates
⍝ a number of global variables that are used to
⍝ to access data stored in these files. These
⍝ variable contain '∆_' in their names.
 ⎕error (1∊'∆_'⍷⎕nl 2)/'DBI globals present - erase all DBI globals'

 DBIOpen dbi                        ⍝ open DBI file
 z ← (∨/'∆_'⍷⎕nl 2)⌿⎕nl 2           ⍝ varibles set by DBI open

⍝ we need table name, column names, types, repeat codes

 t ← ⍎(∨/'∆_fty'⍷z)⌿z       ⍝ column types
 n ← ⍎(∨/'∆_fnm'⍷z)⌿z       ⍝ column names prefixed by table
 rf ← ⍎(∨/'∆_fnc'⍷z)⌿z      ⍝ (0≤) indicates a DBI repeated column (numeric matrix)
 z ← ⎕ex z                  ⍝ clear DBI open globals
 z ← ⎕ex ⊃'∆DBIFL' '∆DBIFN'

⍝ correspondence between DBI and SQLite column types - SQLite does
⍝ not distinguish between integer types and ignores all fixed length
⍝ declarations in SQL column declarations this is ideal for APL data
⍝ C=text, I=integer, U=integer, F=real, D=date)
 st ← ('text' 'integer' 'integer' 'real' 'date')['CIUFD'⍳t]
 sn ← (¯1 + (↑n)⍳'∆') ↑ ↑n          ⍝ table name without '∆'
 nrc ← sc ← (n ⍳¨ '∆') ↓¨ n         ⍝ just column names

⍝ expand any repeated numeric columns
 :if ∨/rm ← 0 < rf
    sc ← ⊂¨ sc
    (rm/sc) ← (rm/sc) ,¨¨  ⍕¨¨ ⍳¨ rm/rf
    sc ← ⊃ ,/ sc
    st ← (1⌈rf) / st
 :endif

⍝ SQLite tables require a primary key DBI files
⍝ do not necessarily have a primary key
 ⎕error ((⊂keyn)∊sc)/'(',keyn,') key name occurs in DBI file - use another name'
 sql ← 'create table ',sn,' (',keyn,' integer primary key, '
 sql ← sql , (¯2 ↓ ∊ sc , ' ' , st ,[1.5] ⊂', '), ')'

⍝ return sql, table name, SQLite types, repeating and non-repeating columns
 z ← sql sn st sc nrc
\end{verbatim}

\vspace{1cm}


\begin{minipage}{0.75\linewidth}

\Large

The UTF8 APL ``trouble makers'' properly handled within
a \verb|lstlisting| environment. Characters following
the APL comment ``lamp'' character   
\verb|⍝| get comment coloring. Syntax coloring
is one of the features of \verb|lstlisting|.

\centering

\begin{lstlisting}[language=apl,extendedchars=true,
basicstyle=\ttfamily\Huge\color{red},frame=shadowbox,
rulesepcolor=\color{blue},framexleftmargin=10mm]
!*+-/<=>?\^|~¨¯×÷←↑→                              
↓∆∇∊∘∨∩∪∵∼≠≡≢≤≥⊂⊃⊖⊢⊣
⊤⊥⋄⌈⌊⌶⌷⌹⌻⌽⌿⍀⍂⍇⍈⍉⍋⍎⍐⍒
⍕⍗⍙⍝⍞⍟⍡⍢⍣⍤⍥⍨⍪⍫⍬⍱⍲⍳⍴⍵
⍷⍸⍺⎕▯◊○₧
\end{lstlisting}
\end{minipage}

\vspace{1cm}

\texttt{lstlisting} verbatim with extended UTF8 APL characters.

\begin{lstlisting}[language=apl,extendedchars=true]

z ← keyn SQLiteFrDBI dbi;t;n;st;sn;sc;sql;rf;rm;nrc;⎕io

⍝ Returns a SQLite table creation statement that maps the
⍝ structure of APL+WIN inverted DBI files to SQLite tables.
⍝
⍝ monad: ev ← SQLiteFrDBI cvDBIFileName
⍝
⍝   sqlv ← SQLiteFrDBI 'C:\BCA\bcadev\CA\ULTCL.DBI'

 ⎕io ← 1

⍝ default SQLite primary key name
 :if 0=⎕nc 'keyn' ⋄ keyn ← 'rwkey' ⋄ :endif

⍝ open DBI file - the DBI open function creates
⍝ a number of global variables that are used to
⍝ to access data stored in these files. These
⍝ variable contain '∆_' in their names.
 ⎕error (1∊'∆_'⍷⎕nl 2)/'DBI globals present - erase all DBI globals'

 DBIOpen dbi                        ⍝ open DBI file
 z ← (∨/'∆_'⍷⎕nl 2)⌿⎕nl 2           ⍝ varibles set by DBI open

⍝ we need table name, column names, types, repeat codes

 t ← ⍎(∨/'∆_fty'⍷z)⌿z       ⍝ column types
 n ← ⍎(∨/'∆_fnm'⍷z)⌿z       ⍝ column names prefixed by table
 rf ← ⍎(∨/'∆_fnc'⍷z)⌿z      ⍝ (0≤) indicates a DBI repeated column (numeric matrix)
 z ← ⎕ex z                  ⍝ clear DBI open globals
 z ← ⎕ex ⊃'∆DBIFL' '∆DBIFN'

⍝ correspondence between DBI and SQLite column types - SQLite does
⍝ not distinguish between integer types and ignores all fixed length
⍝ declarations in SQL column declarations this is ideal for APL data
⍝ C=text, I=integer, U=integer, F=real, D=date)
 st ← ('text' 'integer' 'integer' 'real' 'date')['CIUFD'⍳t]
 sn ← (¯1 + (↑n)⍳'∆') ↑ ↑n          ⍝ table name without '∆'
 nrc ← sc ← (n ⍳¨ '∆') ↓¨ n         ⍝ just column names

⍝ expand any repeated numeric columns
 :if ∨/rm ← 0 < rf
    sc ← ⊂¨ sc
    (rm/sc) ← (rm/sc) ,¨¨  ⍕¨¨ ⍳¨ rm/rf
    sc ← ⊃ ,/ sc
    st ← (1⌈rf) / st
 :endif

⍝ SQLite tables require a primary key DBI files
⍝ do not necessarily have a primary key
 ⎕error ((⊂keyn)∊sc)/'(',keyn,') key name occurs in DBI file - use another name'
 sql ← 'create table ',sn,' (',keyn,' integer primary key, '
 sql ← sql , (¯2 ↓ ∊ sc , ' ' , st ,[1.5] ⊂', '), ')'

⍝ return sql, table name, SQLite types, repeating and non-repeating columns
 z ← sql sn st sc nrc
\end{lstlisting}

\end{document}